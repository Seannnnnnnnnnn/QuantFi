\documentclass{article}
\usepackage{amssymb}
\usepackage{amsthm}
% Language setting
% Replace `english' with e.g. `spanish' to change the document language
\usepackage[english]{babel}

% Set page size and margins
% Replace `letterpaper' with`a4paper' for UK/EU standard size
\usepackage[letterpaper,top=2cm,bottom=2cm,left=3cm,right=3cm,marginparwidth=1.75cm]{geometry}

% Useful packages
\usepackage{amsmath}
\usepackage{graphicx}
\usepackage[colorlinks=true, allcolors=blue]{hyperref}

\title{Forwards \& Futures}
\author{Sean Conlon}

\begin{document}
\maketitle

\section{Forwards}


\section{Introduction}
We begin by outlining the assumptions (and their respective justifications) we make when modelling financial derivatives. \begin{enumerate}
    \item There is a riskless investment, that grows at a continuously and compounded rate $r$. This is a common assumption of all finance and typically approximated by the rate of return on highly liquid government bonds. Traditionally, a bank account with a fixed interest also acts as riskless investment. The predicates of \textit{continuous} and \textit{compounding} return $r$ are mostly technical conditions, as interest is computed at most daily. This means that if $M_t$ is invested in this riskless investment at time $t$, then at time $T>t$
    $$M_T = M_t e^{r(T-t)}$$
    \item Borrowing and lending rates are the same. This is seldom true in practice, for large financial institutions, is \textit{approximately true}. 
    \item There are no transaction costs. Any market participant pays the same to buy or sell the same volume of any asset. 
    \item Assets are infinitely divisible. This is a technical assumption to allow us to work with real numbers. At scale, this is also \textit{approximately true}, as any decimals can just be discretisation errors. 
    \item Short selling is allowed - one can hold any negative amount of an asset. This is a mostly true assumption, though there are restrictions on the maximum amount of any asset that can be shorted. 
\end{enumerate} 


\end{document}