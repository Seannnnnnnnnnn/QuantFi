\documentclass{article}
\usepackage{amssymb}
\usepackage{amsthm}
% Language setting
% Replace `english' with e.g. `spanish' to change the document language
\usepackage[english]{babel}

% Set page size and margins
% Replace `letterpaper' with`a4paper' for UK/EU standard size
\usepackage[letterpaper,top=2cm,bottom=2cm,left=3cm,right=3cm,marginparwidth=1.75cm]{geometry}

% Useful packages
\usepackage{amsmath}
\usepackage{graphicx}
\usepackage[colorlinks=true, allcolors=blue]{hyperref}

\title{Swaps \& Fixed Income Securities}
\author{Sean Conlon}

\begin{document}
\maketitle

\newpage
\section{Bonds}


\subsubsection*{Interest Rates}
For simple interest over $n$ years this 
$$FV = PV(1+r\cdot t\cdot df)^n$$
For compounding interest with fixed rate $r$ over $n$ years with $m$ compounding periods per year
$$FV = PV\left(1+\frac{r}{m}\right)^{nm}$$
For continuous compounding, this occurs when $m\rightarrow\infty$ which results in 
$$FV = PVe^{rn}$$
Continuous compounding is a typical assumption used in derivatives pricing. Given a compounding rate $r_m$ we can find it's \textit{equivalent continuous rate} $r_c$ by equating 
\begin{align*}
    & PVe^{r_cn} = PV\left(1+\frac{r_,}{m}\right)^{nm} \\
    & \therefore r_c = m \ln\left(1+\frac{r_m}{m} \right)
\end{align*}
The $n$\textit{-year zero-coupon rate} or the \textit{n-year zero rate} is the rate of interest earned on an investment that starts today and ends in $n$ years.




\end{document}