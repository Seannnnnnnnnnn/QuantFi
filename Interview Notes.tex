\documentclass{article}
\usepackage{amssymb}
\usepackage{amsthm}
% Language setting
% Replace `english' with e.g. `spanish' to change the document language
\usepackage[english]{babel}

% Set page size and margins
% Replace `letterpaper' with`a4paper' for UK/EU standard size
\usepackage[letterpaper,top=2cm,bottom=2cm,left=3cm,right=3cm,marginparwidth=1.75cm]{geometry}

% Useful packages
\usepackage{amsmath}
\usepackage{graphicx}
\usepackage[colorlinks=true, allcolors=blue]{hyperref}

\title{Interview Notes}
\author{Sean Conlon}

\begin{document}
\maketitle

\newpage
\subsection*{Probability: Coupon Collecting Problem}
Given $n$ coupons each with probability $1/n$ of being drawn, let $T$ be the number of the trials required to see all $n$ coupons. The coupon collecting problem is to find $E(T)$, which turns out to be 
$$E(T) = n H_n = n \sum_{i=1}^{n}\frac{1}{i}$$
\begin{proof}
    Let $t_i$ be the random variable of finding the $i^{th}$ coupon after $i-1$ coupons have been collected. Clearly
    $$t_i \sim \text{geometric}(p_i) \hspace{5mm} \text{where } p_i = \frac{n-(i-1)}{n} = \frac{n-i+1}{n}$$
    So $E(t_i)$ is given by 
    $$E(t_i) = \frac{1}{p_i} = \frac{n}{n-i+1}$$
    Now, we note that $T = t_1 + \dots + t_n$ and thus, by the linearity of expectation: 
    \begin{align*}
        E(T) &= \sum_{i=1}^{n}E(t_i) \\
        &= \left(\frac{n}{n} + \frac{n}{n-1} + \dots \frac{n}{1} \right) \\
        &= n\left(1 + \frac{1}{2} + \dots + \frac{1}{n} \right)
    \end{align*}
\end{proof}
For an example application, consider the following question: \textit{given a fair coin, find the expected number of flips to see both sides}. This is an instance of the coupon collecting problem with $n=2$, and thus the expected number of flips required is $2(1+1/2) = 6$


\subsection*{Combinatorics: Stars \& Bars}


\newpage
\subsection*{Wait Time for $N$ Consecutive Events That Occur With Probability $p$}


\subsection*{Gamblers Ruin}
A useful result for a simple random walk is as follows: Given a simple unit random walk $X$ and a hitting set $\{-a, b\}$ for $a,b>0$ the expected hitting time (expected length of $X$ until the value of the random walk is either $-a$ or $b$) is $-ab$.


\end{document}